% !TEX TS-program = arara
% arara: xelatex: { synctex: on, options: [-halt-on-error] } 
% arara: biber
% arara: biber: { options: [--noremove-tmp-dir] }
% % arara: texindy: { markup: xelatex }
% %% arara: makeglossaries
% arara: xelatex: { synctex: on, options: [-interaction=batchmode, -halt-on-error] }
% arara: xelatex: { synctex: on, options: [-interaction=batchmode, -halt-on-error]  }
% % arara: clean: { extensions: [ aux, log, out, run.xml, ptc, toc, mw, synctex.gz, ] }
% % arara: clean: { extensions: [ bbl, bcf, blg, ] }
% % arara: clean: { extensions: [ glg, glo, gls, ] }
% % arara: clean: { extensions: [ idx, ilg, ind, xdy, ] }
% % arara: clean: { extensions: [ plCode, plData, plMath, plExercise, plNote, plQuote, ] }
%-----------------------------------------------------------------
\documentclass[12pt]{PalisadesLakesBook}
% \geomHDTV
% \geomLandscape
\geomHalfDTV
\geomPortraitOneColumn
%-----------------------------------------------------------------

%\AsanaFonts % misssing \mathhyphen; less on page than Cormorant/Garamond
%\CormorantFonts % light, missing unicode greek
\EBGaramondFonts % fewest pages
%\ErewhonFonts
%\FiraFonts % tall lines, all sans, much less per page, missing \in?
%\GFSNeohellenicFonts 
%\KpFonts
%\LatinModernFonts
%\LegibleFonts
%\LibertinusFonts
%\NewComputerModernFonts
%\STIXTwoFonts
%\BonumFonts % most pages
%\PagellaFonts
%\ScholaFonts
%\TermesFonts
%\XITSFonts

%-----------------------------------------------------------------
\togglefalse{plMath}
\togglefalse{plCode}
\togglefalse{plData}
\togglefalse{plNote}
\togglefalse{plExercise}
\togglefalse{plQuote}
\togglefalse{printglossary}
\togglefalse{printindex}
%-----------------------------------------------------------------
\title{Geology questions}
\author{John Alan McDonald 
(palisades dot lakes at gmail dot com)}
\date{draft of \today}
%-----------------------------------------------------------------
\begin{document}

\maketitle
\PalisadesLakesTableOfContents{7}
%-----------------------------------------------------------------
\begin{plSection}{Introduction}
\end{plSection}%{Introduction}
%-----------------------------------------------------------------
\begin{plSection}{Earth structure}
  \begin{itemize}
    \item Crust-mantle vs lithosphere-asthenosphere?
    \item Upper mantle within lithosphere?
    \item (\citeAuthorYearTitle[chapter 17, Recycling of crustal materials and mantle
heterogeneity]{Anderson:2007:NewTheory})
    $5$ reservoir mixing model for isotope variation in MORB, OIB, and  CRB.
    What's the dimension of the data?
    Is this trivial, or is there some real dimension reduction?
    Does it actually fit in a $4$d simplex?
  \end{itemize}
\end{plSection}%{Earth structure}
%-----------------------------------------------------------------
\begin{plSection}{Plate tectonics}
  \begin{itemize}
    \item Are plates real?
    \item Are ocean and continent plates the same thing?
    \item Just crust or all of lithosphere?
    \item Where are subducting plates/slabs relative to lithosphere/asthenosphere?
    \item Why do plates move?
    \item Is plate motion rigid?
    \item Plate motion relative to what?
    \item Asphericity effects? Could this be a driver?
  \end{itemize}
\end{plSection}%{Plate tectonics}
%-----------------------------------------------------------------

\begin{plSection}{Minerals}
  \begin{itemize}
    \item Is there some rational organizing principle?
    \item Is there a way of plotting mineral attributes that clarifies groups?
    \item Better naming convention, separating rocks and minerals, displaying compositions?
  \end{itemize}

\end{plSection}%{Minerals}
%-----------------------------------------------------------------
\begin{plSection}{Rocks}
\end{plSection}%{Rocks}
%-----------------------------------------------------------------
%-----------------------------------------------------------------
%\BeginAppendices
%\input{typesetting}
%-----------------------------------------------------------------
\end{document}
%-----------------------------------------------------------------
